\chapter{Fazit}
\label{Fazit}
Durch die Umsetzung einer Demo-Anwendung mit PouchDB, dem Aufsetzen einer selbst gehosteten CouchDB und dem Testen der Funktionen hinsichtlich Synchronisation und Konfliktbehandlung wurden die Schlüsseleigenschaften vollständig abgedeckt. Am Ende ist PouchDB ein Aufsatz für das \emph{CouchDB Replication Protocoll}, welcher diese Synchronisations-API um die Möglichkeit von Offline-Speicherung und Adapter auf andere Datenbanksysteme erweitert. Die Verwendung gestaltet sich dabei weitgehend logisch, wobei die in den etablierten TypeScript-Repositories völlig veralteten Definitionsdateien für Frust und zusätzliche Arbeit sorgten. Die PouchDB-Community könnte in Zukunft durch bessere Pflege dieses Faktors die Verwendung des Pakets in modernen Umgebungen wie Angular/Ionic 2 erheblich erleichtern. Beim Aufsetzen der Codebasis mussten außerdem ... //TODO Probleme beim Aufsetzen//

Bei der Installation und vor allem der Konfiguration der CouchDB sollte viel Wert auf die Absicherung dieser Komponente gelegt werden. Ähnlich wie die berüchtigten, offenen Installationen von MongoDB sind die Standard-Einstellungen nicht geeignet, eine produktive Datenbank zu betreiben. Um die Datenbank ausreichend abzusichern muss zunächst der CouchDB-interne Webserver mit einem Reverse Proxy abgeschirmt werden. Danach muss unbedingt die standardmäßig nicht aktivierte Benutzerauthentifizierung sowohl für die einzelnen Datenbanken als auch für die Administrationsoberfläche aktiviert werden.

Sind das Anwendungssetup und die Einrichtung des Backends überstanden, ist die Entwicklung mit PouchDB unkompliziert. Tests auf verschiedenen Endgeräten und im Browser ergaben außerdem, dass die versprochene Funktionalität zuverlässig und von den verschiedenen Plattformen unbeeindruckt zu Verfügung steht. Die wichtigsten Versprechen werden damit gehalten.
Nach Betrachtung der Alternativen sollte bei der Entscheidung über die für ein neues Projekt verwendete Technologie aber bedacht werden, dass man sich durch die Verwendung von PouchDB neben der Implementierung der Applikation auch die Verantwortung für das Backend auferlegt. Zwei der betrachteten Alternativen lagern diese Aufgaben an den Cloudprovider aus. Auch bei Betrachtung der hinter dem Projekt beziehungsweise hinter den Alternativen stehenden Communities/Firmen bleibt festzuhalten, dass PouchDB als rein Community-getriebenes Projekt die vermeintlich geringste Stabilität und Zukunftssicherheit gegenüber den unter anderem von Google oder Microsoft entwickelten Lösungen bietet.
