\chapter{PouchDB}
\label{PouchDB}

\section{Adapter}
\label{Adapter}

Laut eigener Dokumentation ist PouchDB keine eigenständige Datenbank. Stattdessen versteht sich PouchDB als Datenbankabstraktionsschicht und nutzt im Hintergrund andere Datenbanken. Der Zugriff auf diese Datenbanken wird durch sogenannte Adapter umgesetzt. Ziel dieser Abstraktion ist es eine einheitliche Programmierschnittstelle (API) zur Verfügung zu stellen, die in verschiedenen JavaScript-Umgebungen und Browsern gleich funktioniert.

PouchDB wird mit Adaptern für die Browser-Datenbanken IndexedDB und WebSQL sowie einem Adapter für LevelDB ausgeliefert. Weitere Adapter für SQLite, den Localstorage oder eine In-Memory-Datenbank stehen als Plugins zur Verfügung \cite{pouch:adapters}.


Bei dem in dieser Arbeit vorgestellten Technologiestack werden Apps im Browser von mobilen Geräten ausgeführt, deshalb soll der Einsatz von PouchDB im Browser genauer betrachtet werden.

Erzeugt man, wie im folgenden Codeausschnitt beschrieben, eine neue PouchDB-Datenbank, so  nutzt PouchDB im Browser vorzugsweise den IndexedDB-Adapter. Unterstützt der verwendete Browser das nicht, greift PouchDB automatisch auf den WebSQL-Adapter zurück \cite{pouch:adapters}.
\begin{codebox}
\begin{lstlisting}[style=typescript]
var pouch = new PouchDB('myDB');
\end{lstlisting}
\end{codebox}


Die Datenmengen, die in den Browserdatenbanken IndexedDB und WebSQL abgelegt werden können, sind in mobilen Browsern begrenzt \cite{html5:quota}. Um diese Begrenzungen zu umgehen empfiehlt die PouchDB-Dokumentation die Verwendung einer SQLite-Datenbank mit dem entsprechenden Adapter. Eine SQLite-Datenbank kann laut Dokumentation auch den Entwicklungsprozess erleichtern, da SQLite-Datenbanken als .sqllite-Dateien für Entwickler direkt zugänglich sind \cite{pouch:adapters}.

Im folgenden Codeausschnitt wird eine neue PouchDB-Datenbank mit dem Namen \texttt{db} erzeugt und manuell die Verwendung des SQLite-Adapter festgelegt \cite{pouch:adapters}.
\begin{codebox}
\begin{lstlisting}[style=typescript]
var db = new PouchDB('db', {adapter: 'cordova-sqlite'});
\end{lstlisting}
\end{codebox}


\section{API}
\label{API}
PouchDB kann als JavaScript-Implementierung von CouchDB angesehen werden. Das selbst erklärte Ziel von PouchDB ist es die CouchDB-API im Browser und in Node.js zu emulieren \cite{pouch:intro}. 

Die API von PouchDB ist laut eigener Dokumentation asynchron. Sie kann über Callbacks, Promises oder async-Funktionen angesprochen werden \cite{pouch:api}.


\section{Synchronisation}
\label{Synchronisation}
\begin{citeenv}
\begin{quotation}
"`The way I like to think about CouchDB is this: CouchDB is bad at everything, except syncing. And it turns out that's the most important feature you could ever ask for, for many types of software."' \cite{pouch:replication}.
\end{quotation}
\end{citeenv}

Das Zitat, dass Jason Smith, einem Entwickler von CouchDB, zugeordnet wird, deutet an, dass Synchronisation eine wichtige Anforderung an viele verschiedene Arten von Software darstellt.

PouchDB versucht diese Anforderung umzusetzen, in dem sie das CouchDB Replication Protocol implementiert. Dabei handelt es sich um ein in \cite{apache:replication} dokumentiertes Regelwerk, um zwischen Datenbanken, die dieses Regelwerk implementieren, Daten auszutauschen. Laut PouchDB-Dokumentation verwendet das CouchDB Replication Protocol dabei HTTP und eine REST-Schnittstelle zur Kommunikation zwischen den Datenbanken. Das ermöglicht es auch mit einem Browser oder einem REST-Client mit einer Datenbank, die das CouchDB Replication Protocol implementiert, zu kommunizieren \cite{pouch:intro}. Die technischen Details der Synchronisation werden in Kapitel \ref{Konfliktbehandlung} behandelt.


\subsection{Umsetzung in der API}
Die PouchDB-API unterscheidet zwischen der Replikation und der Synchronisation von Datenbanken. Als Replikation wird der unidirektionale Austausch von Daten bezeichnet. Dafür stellt die API die \texttt{replicate}-Methode zur Verfügung. Im folgenden Codebeispiel wird diese genutzt, um von einer lokale PouchDB zu einer entfernten PouchDB Daten zu replizieren \cite{pouch:replication}:
\begin{codebox}
\begin{lstlisting}[style=typescript]
var localDB = new PouchDB('local');
var remoteDB = new PouchDB('http://localhost:5984/remote');
localDB.replicate.to(remoteDB);
\end{lstlisting}
\end{codebox}


Bei der Synchronisation handelt es sich um einen bidirektionale Replikation. Hierfür steht die \texttt{sync}-Methode zur Verfügung. Sie kann wie folgt verwendet werden \cite{pouch:replication}:
\begin{codebox}
\begin{lstlisting}[style=typescript]
localDB.sync(remoteDB);
\end{lstlisting}
\end{codebox}

\subsection{Kontinuierliche Synchronisation}
Die in den bisherigen Codebeispielen gezeigte Synchronisation zwischen zwei Datenbanken ist einmalig. PouchDB unterstützt darüber hinaus eine kontinuierliche Synchronisation, die bei jeder Änderung den Synchronisationsprozess automatisch anstößt. Dies ist in der API über die Option \texttt{live} umgesetzt.

Wie in der Motivation dieser Arbeit beschrieben, muss damit gerechnet werden, dass die Internetverbindung auf mobilen Geräten abreißen kann. Um dem entgegenzuwirken bietet die Synchronisation die Option \texttt{retry} an. Damit kann angegeben werden, dass bei einem Verbindungsabbruch versucht werden soll die Verbindung wiederherzustellen. Der folgenden Codeausschnitt zeigt die Verwendung der \texttt{live}- und der \texttt{retry}-Option \cite{pouch:replication}.
\begin{codebox}
\begin{lstlisting}[style=typescript]
localDB.sync(remoteDB, {live: true, retry: true});
\end{lstlisting}
\end{codebox}

Über die Option \texttt{back\_off\_function} kann konfiguriert werden in welchem Intervall PouchDB nach einem Verbindungsabbruch versucht die Verbindung wieder aufzubauen. Die Dokumentation demonstriert dies an folgendem Codebeispiel \cite{pouch:api}:

\begin{codebox}
\begin{lstlisting}[style=typescript]
localDB.replicate.to(remoteDB, {
  live: true,
  retry: true,
  back_off_function: function (delay) {
    if (delay === 0) {
      return 1000;
    }
    return delay * 3;
  }
});
\end{lstlisting}
\end{codebox}

\section{Integration in den Entwicklungsprozess}

PouchDB wird nicht in TypeScript sondern in JavaScript entwickelt. Daher werden mit dem PouchDB-Paket keine TypeScript-Definitionsdateien ausgeliefert. Im DefinitlyTyped-Repository, das sich als Quelle für Definitionsdateien etabliert hat [Citation Needed], finden sich entsprechende Definitionsdateien, diese entsprechen aber nicht dem aktuellen Entwicklungsstand. Um eine nahtlose Integration in den in Kapitel \ref{Entwicklungsprozess} beschriebenen Entwicklungsprozess zu ermöglichen, mussten die bestehenden Definitionsdateien erweitert werden. 